\input{header}

\AtBeginSubsection[]
{
	\begin{frame}<beamer>
		\frametitle{Outline}
		\tableofcontents[current,currentsubsection]
	\end{frame}
}

\begin{document}

\begin{frame}[allowframebreaks] \frametitle{Chomsky normal form (CNF)}
  \begin{itemize}
\item Purpose: a simplified form of grammars
\item Every rule must be either
  \begin{equation*}
A \rightarrow BC
\end{equation*}
or
\begin{equation*}
A \rightarrow a
\end{equation*}
\item $B,C$ are not start variables
\item $a \in\Sigma $ so
  \begin{equation*}
    A \rightarrow \epsilon
  \end{equation*}
  is not allowed. 
\item However,
  \begin{equation*}
S \rightarrow \epsilon
\end{equation*}
is allowed, where $S$ is the start
variable
\item This form is useful later (but not in this chapter)
\item To convert a CFG to a CNF, let's show an example first
\end{itemize}\end{frame}

\begin{frame}[allowframebreaks] \frametitle{Example to convert CFG to CNF}
  \begin{itemize}
\item The original CFG
  \begin{eqnarray*}
&& S \rightarrow ASA\mid aB\\
&& A \rightarrow B\mid S\\
&& B \rightarrow b \mid \epsilon\\
\end{eqnarray*}
\item Add
  \begin{equation*}
S_0 \rightarrow S
\end{equation*}
because the start variable cannot be on the right
  \begin{eqnarray*}
&& S_0 \rightarrow S\\
&& S \rightarrow ASA\mid aB\\
&& A \rightarrow B\mid S\\
&& B \rightarrow b \mid \epsilon\\
\end{eqnarray*}
\item Remove
  \begin{equation*}
  B \rightarrow \epsilon 
\end{equation*}
because $\epsilon$ cannot be on the right
\begin{eqnarray*}
&& S_0 \rightarrow S\\
&& S \rightarrow ASA\mid aB|\alert{a}\\
&& A \rightarrow B\mid \alert{\epsilon} \mid S\\
&& B \rightarrow b\\
\end{eqnarray*}
\item Remove $A\rightarrow \epsilon$ because the right-hand side
  cannot be a single variable
\begin{eqnarray*}
&& S_0 \rightarrow S\\
&& S \rightarrow ASA\mid aB\mid a \mid \alert{AS\mid SA \mid S}\\
&& A \rightarrow B\mid S\\
&& B \rightarrow b
\end{eqnarray*}
\item What if
  \begin{equation*}
    B \rightarrow \epsilon
  \end{equation*}
appears again? An infinite loop? We will discuss this issue later
\item Remove $S\rightarrow S$
\begin{eqnarray*}
&& S_0 \rightarrow S\\
&& S \rightarrow ASA\mid aB\mid a \mid AS\mid SA\\
&& A \rightarrow B\mid S\\
&& B \rightarrow b
\end{eqnarray*}
 
\item Remove $S_0 \rightarrow S$
  \begin{eqnarray*}
&& S_0 \rightarrow \alert{ASA\mid aB\mid a \mid AS\mid SA}\\
&& S \rightarrow ASA\mid aB\mid a \mid AS \mid SA\\
&& A \rightarrow B\mid S\\
&& B \rightarrow b
\end{eqnarray*}

\item Remove $A\rightarrow B, A\rightarrow S$
  \begin{eqnarray*}
&& S_0 \rightarrow ASA\mid aB\mid a \mid AS\mid SA\\
&& S \rightarrow ASA\mid aB\mid a \mid AS \mid SA\\
&& A \rightarrow \alert{b \mid ASA \mid aB\mid a \mid AS \mid SA}\\
&& B \rightarrow b
\end{eqnarray*}
\item Finally
\begin{eqnarray*}
&& S_0 \rightarrow AA_1\mid UB\mid a \mid AS\mid SA\\
&& S \rightarrow AA_1 \mid UB\mid  a \mid AS \mid SA\\
&& A \rightarrow b \mid AA_1 \mid UB\mid a \mid AS \mid SA\\
&& A_1 \rightarrow SA\\
&& U \rightarrow a\\
&& B \rightarrow b\\
  \end{eqnarray*}

\end{itemize}\end{frame}

\begin{frame}[allowframebreaks] \frametitle{Convert Context-free to Chomsky normal}

  A procedure to summarize what we have done in the example
  
  \begin{itemize}
\item $S_0 \rightarrow S$

\item [] So start state not on the right
\item Remove $A \rightarrow \epsilon$, where $A$ is not the start state:
\item[] For any rule of
  \begin{equation*}
\cdots  \rightarrow uAv 
\end{equation*}
 add
 \begin{equation*}
\cdots \rightarrow uv
\end{equation*}
\item Remove
  \begin{equation*}
  A \rightarrow B
\end{equation*}
because the right hand cannot have a single variable.
For any
\begin{equation*}
  B \rightarrow u
\end{equation*}
we
\begin{center}
remove $A \rightarrow B$ and   $B \rightarrow u$,
and add $A \rightarrow u$
\end{center}
\item After this, we have either
  \begin{equation*}
    A \rightarrow u_1 \cdots u_k, u_i \in V
      \mbox{ or } \Sigma;
\end{equation*}
and 
\begin{equation*}
\mbox{ if }
k = 1, \mbox{ then } u_i \in \Sigma
\end{equation*}
\item Replace the right side with
  \begin{equation*}
    \begin{split}
& A \rightarrow u_1A_1 \\
& A_1 \rightarrow u_2A_2 \\
& \cdots
\end{split}
\end{equation*}
\item Finally add
  \begin{equation*}
U_i \rightarrow u_i \text{ if } u_i \in \Sigma    
  \end{equation*}
\end{itemize}\end{frame}

 \begin{frame}[allowframebreaks] \frametitle{Infinite loop in the above procedure}

  \begin{itemize}
\item Original rules
  \begin{eqnarray*}
    && S \rightarrow B \mid \epsilon\\
&& B \rightarrow S \mid \epsilon
  \end{eqnarray*}

\item Add $S_0$
  \begin{eqnarray*}
&& S_0 \rightarrow S\\
&& S \rightarrow B \mid \epsilon\\
&& B \rightarrow S \mid \epsilon
  \end{eqnarray*}

\item Remove $S \rightarrow \epsilon$
  \begin{eqnarray*}
&& S_0 \rightarrow S \mid \epsilon\\
&& S \rightarrow B \\
&& B \rightarrow S \mid \epsilon
  \end{eqnarray*}

\item Remove $B \rightarrow \epsilon$
  \begin{eqnarray*}
&& S_0 \rightarrow S \mid \epsilon\\
&& S \rightarrow B \mid \alert{\epsilon} \\
&& B \rightarrow S 
  \end{eqnarray*}

\item No need to add $S \rightarrow \epsilon$
\item Reason: $S \rightarrow \epsilon$ has been
handled; see line -8 of p109 in the textbook.

\end{itemize}\end{frame}


\end{document}
%%% Local Variables:
%%% mode: latex
%%% TeX-master: t
%%% End:
