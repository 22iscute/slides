\input{header}

\AtBeginSubsection[]
{
	\begin{frame}<beamer>
		\frametitle{Outline}
		\tableofcontents[current,currentsubsection]
	\end{frame}
}

\begin{document}

\begin{frame}[allowframebreaks] \frametitle{NP $\equiv$ Polynomial NTM}

  \begin{itemize}
\item Polynomial verifier $\Leftrightarrow$ polynomial NTM

\item Idea:

\item [] ``$\Rightarrow$'' NTM by guessing certificate

\item [] ``$\Leftarrow$'' using NTM's accepting branch as  certificate
\item Proof:
\item ``$\Rightarrow$'': now we have a verifier V in time $O(n^k)$

  
\item [] Recall the definition below
    \begin{equation*}
    A=\{w\mid
V \mbox{ accepts } 
\langle  w,c\rangle  
\mbox{ for some strings } c\}
  \end{equation*}
We have
\begin{equation*}
|c|\leq n^k
\end{equation*}
\item Use an NTM to
  \begin{enumerate}
  \item nondeterministically select $c$
  \item run V on $\langle  w,c\rangle $
  \end{enumerate}
That is, run $c$ in parallel and each is polynomial
\item We have that for any $w \in A$, the NTM accepts
  it in polynomial time
\item ``$\Leftarrow$'': now $w$ is accepted by a polynomial NTM
\item [] Let $c$ be the accepting branch
\item [] Note that for polynomial NTM, each branch is polynomial
\item Then we have a verifier V that handles input $\langle  w,c\rangle $
in polynomial time

\item Note: the definition of V requires only ``some c.''
\item So finding one is sufficient

\end{itemize}\end{frame}

% \begin{frame}[allowframebreaks] \frametitle{Clique Problem}
%   \begin{itemize}
% \item clique: a fully subgraph connected

% Fig 7.23

% clique: means a small group of people

% \item $k$-clique: $k$ \# nodes
% \item problem: whether $G$ has a clique of given \# nodes
% \end{itemize}\end{frame} \begin{frame}[allowframebreaks] \frametitle{CLIQUE is NP}
%   \begin{itemize}
% \item try to identify some certificate

% clique is $c$
% \item $V$: input $\langle  \langle  G,k\rangle ,c\rangle $

% $\langle  G,k\rangle $ given and some claims $c$ is a $k$-clique. Verify it
%   \begin{enumerate}
%   \item \# nodes of $c$ is $k$ ?
%   \item $c$ fully connected and edges in $G$
%   \end{enumerate}
%   \item $V$ is polynomial
%   \item Easily think from NPTM

% (omitted here)



% \end{itemize}\end{frame}

\begin{frame}[allowframebreaks] \frametitle{SUBSET-SUM}
  \begin{itemize}
\item Given $x_1, \ldots, x_k$ and $t$, is sum of a subset $=t$?
\item Formally
  \begin{gather*}
   \{\langle  s,t\rangle 
\mid s=\{x_1, \ldots, x_k\}
\mbox{ and }
\exists \\
\{y_1, \ldots, y_l\}
\subset \{x_1, \ldots, x_k\} \text{ such that }
\sum y_i =t\}
\end{gather*}
\item Example
  \begin{equation*}
\langle  \{4,11,16,21,27\},25\rangle  \text{ OK as } 4+21=25
\end{equation*}
\item Note: allow repetition here
  \begin{equation*}
\langle  \{4,11,11,16,21,27\},25\rangle 
\end{equation*}
\item We prove that this problem is NP
\item Consider any input
  \begin{equation*}
  \langle  \langle  s,t\rangle ,c\rangle 
\end{equation*}
we 
  \begin{enumerate}
  \item check if $\sum c_i=t$
  \item check if all $c_i \in S$
  \end{enumerate}
\item Here
  \begin{equation*}
  \text{length of } c < \text{  length of } s
\end{equation*}
\item The verification can be done in polynomial time
\end{itemize}
\end{frame}

% \begin{frame}[allowframebreaks] \frametitle{Complements}
%   \begin{itemize}
% \item $\overline{CLIQUE}$,
% $\overline{SUBSET-SUM}$ ? NP

% not that easy

% \item coNP: languages whose complements are NP

% ?coNP = NP, not known
% \end{itemize}\end{frame}

\begin{frame}[allowframebreaks] \frametitle{P vs. NP}
  \begin{itemize}
\item Roughly

\item [] P: problems decided quickly

\item [] NP: problems verified quickly
\item Question: is P = NP?

\item [] This is one of the greatest unsolved problems

\item Most believe P $\neq $ NP
\end{itemize}\end{frame} \begin{frame}[allowframebreaks] \frametitle{NP-completeness}
  \begin{itemize}
\item It has been shown that some problems in NP are related

\item For certain NP problems:

\item [] If $\exists$ a polynomial algorithm of one NP $\Rightarrow$ P = NP

\item These probelsm are called NP-complete problems
\item They are useful to study the issue of P versus NP
\item To prove P $\neq$ NP: only need to focus on
NP-complete problems
\item To prove P=NP: need only polynomial algorithms
for an NP-complete problem
\end{itemize}
\end{frame}
\end{document}
