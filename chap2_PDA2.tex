\input{header}

\AtBeginSubsection[]
{
	\begin{frame}<beamer>
		\frametitle{Outline}
		\tableofcontents[current,currentsubsection]
	\end{frame}
}

\begin{document}


\begin{frame}[allowframebreaks]
  \frametitle{PDA $\rightarrow$ CFL}
  \begin{itemize}
\item Lemma 2.27
\item Language recognized by PDA
$\Rightarrow$ context free
\item Idea:
  \begin{equation*}
    \begin{split}
& \text{ any states } p,q \text{ of a PDA P} \\
\Rightarrow & \text{ we have a variable } A_{pq}
\end{split}
\end{equation*}
and
\begin{eqnarray}
&& \mbox{$A_{pq}$ generates $x
\Leftrightarrow$}  \label{eq:emptystack} \\ 
&& \mbox{P from $p$ with empty
stack to $q$ with empty stack}
\nonumber
\end{eqnarray}
\item Need to modify P so that

  \begin{enumerate}
  \item Single accept: $q_{accept}$

  \item [] Then  $A_{q_{start} q_{accept}}$
    is the start state to generate any string $x$
    of this language
  \item Stack should be empty before accepting
  \item [] In the beginning stack is empty and we need this property to
    have \eqref{eq:emptystack}
  \item Each transition push or pop, but not both
  \end{enumerate}

\item We will explain how to make the PDA satisfy these conditions

\item Now we focus on the more important part: construction of the rules

\item For \eqref{eq:emptystack} we don't really mean
  ``empty stack.'' We actually mean ``stack with the same
  contents.''
\item For the following figure, rule
  \begin{equation*}
    A_{pq} \rightarrow A_{pr} A_{rq}, \forall p, q, r \in Q
  \end{equation*}
should be generated
\item $x$-axis: input string
\item [] $y$-axis: stack height
  
\begin{center}
\begin{tikzpicture}
\draw[] (-0.2,0) -- (6,0);
\draw[] (0,-0.2) -- (0,3.5);
\draw[-] (0,0) .. controls (2,3.5) .. (3.5, 0);
\draw[-] (3.5, 0) .. controls (3.8, 2) .. (5.5,0);
\draw[draw=none] (0,-0.3) node {$p$};
\draw[draw=none] (3.5,-0.3) node {$q$};
\draw[draw=none] (5.5,-0.3) node {$r$};
\end{tikzpicture}
\end{center}

\item We have
  \begin{equation*}
  p,q,r,s \in Q, t \in \Gamma, a, b
\in \Sigma_\epsilon
\end{equation*}
If
\begin{equation*}
(r,t)\in \delta(p,a,\epsilon),
(q,\epsilon)\in \delta(s,b,t)
\end{equation*}
then we should have
\begin{equation*}
A_{pq}\rightarrow aA_{rs} b
\end{equation*}

\begin{center}
\begin{tikzpicture}
\draw[] (-0.2,0) -- (6,0);
\draw[] (0,-0.2) -- (0,3.5);
\draw[-] plot [smooth] coordinates { (0,0) (0.55,1) (2,3.2) (3.1, 1) (4.2, 2.6) (5.05, 1 ) (5.5,0)};
\draw[-,dashed] (0.55,1) -- (0.55,0);
\draw[-,dashed] (5.05,1) -- (5.05,0);
\draw[draw=none] (-0.2,-0.3) node {$p$};
\draw[draw=none] (0.3,-0.3) node {$a$};
\draw[draw=none] (5.3,-0.3) node {$b$};
\draw[draw=none] (5.7,-0.3) node {$q$};
\draw[draw=none] (0.8,1) node {$r$};
\draw[draw=none] (4.8,1) node {$s$};
\end{tikzpicture}
\end{center}
\item Finally we need
  \begin{equation*}
A_{pp}\rightarrow \epsilon, \forall p \in Q
  \end{equation*}

\item Let's discuss an example first

\end{itemize}\end{frame} \begin{frame}[allowframebreaks] \frametitle{Examples}
  \begin{itemize}
\item $\{0^n1^n\mid n \geq \alert{1}\}$
\item Figure 2.15 but $q_1$ not an accept state

\begin{center}
\begin{tikzpicture}
\node[state,initial] (q_1) {$q_1$};
\node[state] (q_2) [right of=q_1] {$q_2$};
\node[state] (q_3) [below of=q_2] {$q_{3}$};
\node[state,accepting] (q_4) [left of=q_3] {$q_{4}$};      
  \path 
  (q_1) edge[above]  node {$\epsilon, \epsilon \rightarrow \$ $} (q_2)
  (q_2) edge[loop right]  node {$0,\epsilon \rightarrow 0$} (q_2)
  (q_2) edge[right]  node {$1, 0 \rightarrow \epsilon$} (q_3)
  (q_3) edge[loop right]  node {$1, 0 \rightarrow \epsilon$} (q_3)  
  (q_3) edge[below] node {$\epsilon, \$ \rightarrow \epsilon$} (q_4);
  \end{tikzpicture}
\end{center}
  
\item Three conditions satisfied

Each transition push or pop only
\item $t=\$$

  \begin{center}
  \begin{tabular}{lllllll}
p & r & s & q & t & a & b\\ \hline
1 & 2 & 3 & 4 & \$ & $\epsilon$ & $\epsilon$
  \end{tabular}
\end{center}
rule:
\begin{equation*}
A_{14} \rightarrow \epsilon A_{23}\epsilon
\end{equation*}
\item $t = 0$

  \begin{center}
  \begin{tabular}{lllllll}
p & r & s & q & t & a & b\\ \hline
2 & 2 & 2 & 3 & 0 & 0 & 1\\
2 & 2 & 3 & 3 & 0 & 0 & 1\\
  \end{tabular}
\end{center}

rules:
\begin{equation*}
  \begin{split}
& A_{23} \rightarrow 0 A_{22} 1 \\
& A_{23} \rightarrow 0 A_{23} 1
\end{split}
\end{equation*}

\item Other rules: 64 rules
  \begin{eqnarray*}
&&A_{11}\rightarrow A_{11}A_{11}\\
&&A_{11}\rightarrow A_{12}A_{21}\\
&&A_{11}\rightarrow A_{13}A_{31}\\
&&A_{11}\rightarrow A_{14}A_{41}\\
&&\vdots
\end{eqnarray*}
and
\begin{eqnarray*}
&& A_{11}\rightarrow \epsilon \\  
&& A_{22}\rightarrow \epsilon \\
&& A_{33}\rightarrow \epsilon \\
&& A_{44}\rightarrow \epsilon 
\end{eqnarray*}

\end{itemize}\end{frame}

\begin{frame}[allowframebreaks]
  \frametitle{The Overall Procedure}
  \begin{itemize}
  \item Given
    \begin{equation*}
    P=(Q,\Sigma, \Gamma, \delta, q_0, \{q_{accept}\})
  \end{equation*}
  \item Construct a CFG $G$
    \begin{equation*}
\text{var}(G) =\{A_{pq}\mid p, q \in Q\}
\end{equation*}
\item Start variable:
  \begin{equation*}
  A_{q_0, q_{accept}}
\end{equation*}

\item Rules: see earlier slides
  \end{itemize}
\end{frame}

\begin{frame}[allowframebreaks]
  \frametitle{Needed modifications of PDA}
  \begin{itemize}
  \item Recall we need PDA to satisfy
    \begin{enumerate}
    \item Single accept state
    \item Stack empty before accepting
    \item Each transition push or pop, but not both
    \end{enumerate}
  \item Single accept:

  \item [] A new start $q_s \rightarrow q_{s'}$ with $\epsilon, \epsilon
\rightarrow \$ $

\item [] For any $q \in F$, $\epsilon, a\rightarrow \epsilon $ to $q$, $\forall a$ to pop things out
\item [] Then $\epsilon, \$ \rightarrow
\epsilon$ to $q_a$.

\item Original PDA:

  \begin{tikzpicture}
    \node[state,initial] (q_0) {$q_{s'}$};
\node[state,draw=none,fill=white] (q_d) [right of=q_0, xshift=-1cm] {$\cdots$};    
\node[state] (q_1) [above right of=q_d, accepting, yshift=-1cm] {$q_1$};
\node[state] (q_2) [below right of=q_d, accepting, yshift=1cm] {$q_2$};
  \end{tikzpicture}

  New:
  \begin{tikzpicture}
    \node[state] (q_0) {$q_{s'}$};
    \node[state,initial above, left of=q_0] (q_s) {$q_{s}$};    
\node[state,draw=none,fill=white] (q_d) [right of=q_0, xshift=-1.3cm] {$\cdots$};    
\node[state] (q_1) [above right of=q_d, yshift=-1.2cm,xshift=-0.3cm] {$q_1$};
\node[state] (q_2) [below right of=q_d, yshift=1.2cm,xshift=-0.3cm] {$q_2$};
\node[state,accepting, right of=q_d, xshift=2cm] (q_a) {$q_{a}$};

  \path 
  (q_s) edge[above]  node {$\epsilon, \epsilon \rightarrow \$ $} (q_0)
  (q_1) edge[loop above]  node {$\epsilon, a \rightarrow \epsilon$\quad $\epsilon, b \rightarrow \epsilon$ } (q_1)
  (q_2) edge[loop below]  node {$\epsilon, a \rightarrow \epsilon$\quad $\epsilon, b \rightarrow \epsilon$} (q_2)  
  (q_1) edge[above]  node {$\epsilon, \$ \rightarrow \epsilon$} (q_a)
  (q_2) edge[below]  node {$\epsilon, \$ \rightarrow \epsilon$} (q_a);
  \end{tikzpicture}
\item  To have each transition push or pop, but not both,
change
\begin{center}
$q_1 \rightarrow q_2$ with $a, a \rightarrow b$ 
\end{center}
to
\begin{center}
  \begin{tabular}{l}
$q_1 \rightarrow q_3, a, a \rightarrow \epsilon$\\
$q_3 \rightarrow q_2, \epsilon, \epsilon \rightarrow b$  
  \end{tabular}
\end{center}
and change 
\begin{center}
$q_1\rightarrow q_2, a, \epsilon \rightarrow \epsilon$ 
\end{center}
to
\begin{center}
  \begin{tabular}{l}
$q_1 \rightarrow q_2, a, \epsilon \rightarrow ?$ \\
$q_2 \rightarrow q_3, \epsilon, ? \rightarrow \epsilon$
  \end{tabular}
\end{center}
  \end{itemize}
\end{frame}

% \begin{frame}[allowframebreaks]
%   \frametitle{Claim 2.30, ``$\Rightarrow$'' of (\ref{eq:emptystack})}
%   \begin{itemize}  
% \item $A_{pq}$ generates $x \Rightarrow p \rightarrow q$ with
% empty stack
% \item formal proof by induction
% on \# steps to derive $x$

% basis: 1-step derivation
% \begin{equation*}
%   A_{pp} \rightarrow \epsilon
% \end{equation*}

% is our only such rule

% trivial: $p$ with empty stack to $p$ with empty stack

% induction: ok rules with derivation length
% $\leq k$

% $A\stackrel{*}{\Rightarrow} x, k+1$ steps

% ($\stackrel{*}{\Rightarrow}$ defined on p 94)
% \item first step
%   \begin{equation*}
%     A_{pq}\Rightarrow a A_{rs}b
% \mbox{ or } A_{pq}
% \Rightarrow A_{pr} A_{rq}
%   \end{equation*}
%   \begin{enumerate}
%   \item $x=ayb,p\rightarrow r, $ stack $\{\$\}\rightarrow\{a\}$

% induction $r\rightarrow s$ with $y$ and stack $\{a\}$

% $s\rightarrow q$, stack $\{a\}\rightarrow \{\$\}$
% \item $x=yz$

% $A_{pr}\stackrel{*}{\Rightarrow}y,
% A_{rq}\stackrel{*}{\Rightarrow}z\leq k$ steps

%   \end{enumerate}
% \end{itemize}\end{frame} \begin{frame}[allowframebreaks] \frametitle{Claim 2.31: ``$\Leftarrow$'' of (\ref{eq:emptystack})}
%   \begin{itemize}
% \item ($p$ empty stack)
% $\stackrel{x}{\rightarrow}$
% ($q$ empty stack)
% $\Rightarrow A_{pq}$ generates $x$
% \item induction on \#steps by PDA

% basic: 0 step

% $p\rightarrow p$ with $\epsilon$

% $A_{pp}\rightarrow \epsilon$ in G, OK

% induction: $k$ ok $k+1$ ?

% $p\rightarrow q$ with $x$ , empty stack
% \begin{enumerate}
% \item 
% empty, only beginning and end
% \item empty in the middle as well
% \end{enumerate}
% \begin{enumerate}
% \item symbol pushed at $p=$
% popped at $q$

% (not empty in the middle)

% call it $t$

% $p$: input $a$ , $q$: input $b$

% $(r,t)\in \delta(p,a,\epsilon), 
% (q,\epsilon)\in \delta(s,b,t)$

% $\Rightarrow \rightarrow a A_{rs} b$ in G

% $x=ayb$

% by induction $A_{rs}\stackrel{*}{\Rightarrow}y$ so

% $A_{pq}\stackrel{*}{\Rightarrow}x$


% \item 
% Let $r$ a state stack empty

% $x=yz$

% $p\rightarrow r,r\rightarrow q \leq k$ steps

% by induction $A_{pr}\stackrel{*}{\Rightarrow}y,
% A_{rq}\stackrel{*}{\Rightarrow} z$

% our construction $A_{pq}\rightarrow A_{pr}A_{rq}$ in G

% $A_{pq}\stackrel{*}{\Rightarrow} x$
% \end{enumerate}
% \end{itemize}\end{frame}

\begin{frame}[allowframebreaks] \frametitle{Regular language is context Free}
  \begin{itemize}
\item We roughly know this but didn't give a formal proof. Here are the steps
\item Regular language $\Rightarrow$ recognized by DFA (in Chapter 1)
\item DFA is a PDA
\item Thus regular language recognized by PDA
\item Then any regular language is context free (by the proof in this chapter)

\end{itemize}\end{frame} \begin{frame}[allowframebreaks] \frametitle{Non-context free languages}
  \begin{itemize}
\item There are such languages
\item We omit the discussion

\end{itemize}\end{frame}

\end{document}
