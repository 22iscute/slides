\input{header}

\AtBeginSection[]
{
	\begin{frame}<beamer>
		\frametitle{Outline}
		\tableofcontents[current,currentsubsection]
	\end{frame}
}

\begin{document}

\begin{frame}[allowframebreaks] \frametitle{Nonregular language}
We are interested in the limit of finite automata
  \begin{itemize}
\item Some languages cannot be recognized
  \begin{equation*}
    \{0^n 1^n \mid n \geq 0\}
  \end{equation*}
\item We might remember \#0 first

\item [] But \# of possible $n$'s is  $\infty$

\item Thus we cannot recognize it by finite automata

\item However, this is not a formal proof
\item It may be difficult to quickly tell if a language is
  regular or not
\item Consider two languages
  \begin{gather*}
    C=\{w\mid \#0=\#1\}\\
D=\{w \mid \#01=\#10\}
  \end{gather*}
It seems that both are not regular

\item [] Indeed, $C$ is not regular but $D$ is
\item [] This is an exercise in the book, so we don't give details
\item To formally prove a language is not regular, we will
  introduce the pumping lemma
\end{itemize}\end{frame} \begin{frame}[allowframebreaks] \frametitle{Pumping lemma}
  \begin{itemize}
\item Strategy: by contradiction
\item We prove
  \begin{center}
regular $\Rightarrow$ some properties    
  \end{center}
\item If ``some properties'' cannot hold, then the language is not
  regular

\end{itemize}\end{frame} \begin{frame}[allowframebreaks] \frametitle{Theorem 1.70}
  \begin{itemize}
\item If $A$ regular $\Rightarrow
\exists p$ (pumping length) such that
$\forall s \in A, |s| \geq p$, 
$\exists x,y,z$ such that $s = xyz$ and
\begin{enumerate}
\item $\forall i \geq 0, xy^i z \in A$
\item $|y| > 0$
\item $|xy| \leq p$
\end{enumerate}
Note that for $y^i$, we have  $y^0=\epsilon$
\end{itemize}\end{frame}

\begin{frame}[allowframebreaks] \frametitle{Proof of pumping lemma}
\begin{itemize}
\item Because $A$ is regular, $\exists$ a DFA
   to recognize A
 \item [] Let $p$ = \# states of this DFA
 \item If no string $s$ such that
   $|s| \geq p$, then the theorem statement is satisfied

\item Now consider any $s$ with $|s| \geq p$
  \begin{equation*}
    s = s_1 \cdots s_n
  \end{equation*}
\item To process this string, assume the
  state sequence is 
  \begin{equation*}
q_1 \cdots q_{n+1}
\end{equation*}
Because $|s|\geq p$, we have
\begin{equation*}
n+1 > p
\end{equation*}

\item In 
$1\ldots p+1$ two states must be the same (pigeonhole principle)

\item [] Fig 1.72


  \begin{center}
\begin{tikzpicture}
  \node[state, initial] (1) {$q_1$};
  \node[state, above right of=1] (9) {$q_j, q_l$};  
  \node[state, accepting, above right of=1, xshift=3cm, yshift=-1cm] (13) {$q_{a}$};        

  \draw (1) edge[bend left, above,dashed] node{$x$} (9);
  \draw (9) edge[bend right, below, dashed] node{$z$} (13);
  \draw (9) edge[loop above, above, dashed] node{$y$} (9);
  \end{tikzpicture}
\end{center}

\item Assume 
  \begin{equation*}
q_j \text{ and } q_l \text{ with } j \leq p+1, l \leq p+1
\end{equation*}
are two same states. Then let
\begin{equation*}
x = s_1 \cdots s_{j-1},
y = s_j,  \cdots s_{l-1},
z = s_l \cdots s_n
\end{equation*}
We then have
\begin{equation*}
  \forall i \geq 0, xy^i z \in A
\end{equation*}
\item Because $j \neq l$, 
\begin{equation*}
|y| > 0
\end{equation*}
From $l \leq p+1$, we have
\begin{equation*}
|xy| \leq p
\end{equation*}
Thus all conditions are satisfied
\end{itemize}\end{frame} 



\end{document}
