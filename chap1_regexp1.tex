\input{header}

\AtBeginSection[]
{
	\begin{frame}<beamer>
		\frametitle{Outline}
		\tableofcontents[current,currentsubsection]
	\end{frame}
}

\begin{document}

\begin{frame}[allowframebreaks] \frametitle{Regular Expressions}
  \begin{itemize}
\item Example
  \begin{equation*}
    (0 \cup 1) 0^*
  \end{equation*}
\item This is a simplification of 
  \begin{equation*}
    (\{0\}\cup \{1\})\circ \{0\}^*
  \end{equation*}
\item Regular expressions are practically useful
\item [] Example: finding lines in a file containing ``a'' or ``b''

  \begin{alltt}
    egrep '(a|b)' file
  \end{alltt}

\item Formally this means we consider the language
  \begin{equation*}
  \Sigma^* a \Sigma^* \cup
\Sigma^* b \Sigma^*
\end{equation*}
$\Sigma^*$: all strings over $\Sigma$
\item Example
  \begin{equation*}
    (0\cup 1)^*
  \end{equation*}
all strings by 0 and 1
\item Example:
  \begin{equation*}
    (0\Sigma^*)\cup (\Sigma^*1)
  \end{equation*}
all strings start with 0 or end with 1
\end{itemize}\end{frame}
\begin{frame}[allowframebreaks] \frametitle{Formal definition of regular expression}
\begin{itemize}
\item $R$ is a regular expression if it is one of the following
expressions
\begin{enumerate}
\item $a$, where $a \in \Sigma$
\item $\epsilon$ $(\epsilon \notin \Sigma)$
\item $\emptyset$
\item $R_1 \cup R_2$,  where $R_1, R_2$ are regular expressions
\item $R_1 \circ R_2$, where $R_1, R_2$ are regular expressions
\item $R_1^*$,  where $R_1$ is a regular expression
\end{enumerate}
\item  Some notes
\item $\emptyset$ and $\epsilon$

\item  [] $\epsilon$: empty string

\item []  $\emptyset$: empty language (language without any string)

  \begin{equation*}
    \begin{split}
&(0 \cup \epsilon)1^*=01^* \cup 1^* \\
&(0 \cup \emptyset)1^*=01^* \\
&\emptyset 1^* = 1^* \emptyset = \emptyset
\end{split}
\end{equation*}
Concatenating $1^*$ with nothing $\Rightarrow$ nothing

\item We have an inductive definition

\item [] An expression is constructed from smaller strings

\end{itemize}\end{frame}

\begin{frame}[allowframebreaks] \frametitle{Examples}
  \begin{itemize}  
\item $0^*10^*$: strings with exactly one 1 
\item $(\Sigma \Sigma)^*$: strings with even length
\item Assume $\Sigma = \{0,1\}$
  \begin{equation*}
0 \Sigma^*0 \cup 1 \Sigma^*1\cup 0 \cup 1
\end{equation*}
Strings that start
and end with the same symbol
\item $\emptyset^*=\{\epsilon\}$
\item $R\cup \emptyset=R$
\item $R \circ \epsilon=R$
\end{itemize}\end{frame} \begin{frame}[allowframebreaks] \frametitle{Floating number in language}
\begin{equation*}
  (+\cup-\cup\epsilon)(
DD^*\cup DD^*.D^*\cup D^*.DD^*),
\end{equation*}
where
\begin{equation*}
  D = \{0, \ldots,9\}
\end{equation*}
  \begin{itemize}
\item 72, 2.1, 7., -.01
  \begin{equation*}
    \begin{split}
      & 72 \in DD^*\\
      & 2.1 \in DD^*.D^*\\
      &  7. \in DD^*.D^*\\
      & .01 \in D^*.DD^*
    \end{split}
  \end{equation*}

\item Why not $D^*.D^*$

\item [] . is not allowed

\end{itemize}\end{frame} \begin{frame}[allowframebreaks] \frametitle{Equivalence with finite automata}
  \begin{itemize}
\item They have equivalent descriptive power
\item language regular
$\Leftrightarrow$ described by regular expression
\end{itemize}\end{frame} \begin{frame}[allowframebreaks] \frametitle{Lemma 1.55}
  \begin{itemize}
\item Language by a regular expression

$\Rightarrow$ regular (described by an automaton)
\item The proof is by induction. We go through all cases in the
  definition
\item $R = a \in \Sigma$
\item [] Can such a language be recognized by an NFA?
\item [] This language has only one string and can be recognized by

\begin{center}
  \begin{tikzpicture}
\node[state, initial] (q1) {};
\node[state, accepting, right of=q1] (q2) {};

\draw (q1) edge[above] node{$a$} (q2);
  \end{tikzpicture}
\end{center}

\item[] Note that this is an NFA. 
\item[] Formal definition:

  \begin{equation*}
    \begin{split}
& N = (\{q_1,q_2\}, \Sigma, \delta, q_1, \{q_2\}) \\
& \delta(q_1,a)=q_2 \\
& \delta(r,b)=\emptyset, r \neq q_1\mbox{ or } b \neq a
\end{split}
\end{equation*}
\item $R=\epsilon$

\begin{center}
  \begin{tikzpicture}
\node[state, accepting, initial] (q1) {};
  \end{tikzpicture}
\end{center}
  
\item [] Formal definition

  \begin{equation*}
    \begin{split}
& N=(\{q_1\},\Sigma,\delta, q_1, \{q_1\}) \\
& \delta(q_1,a)=\emptyset, \forall a
\end{split}
\end{equation*}

\item $R=\emptyset$

\begin{center}
  \begin{tikzpicture}
\node[state, initial] (q1) {};
  \end{tikzpicture}
\end{center}
  
\item [] Formal definition
\begin{gather*}
  N=(\{q\}, \Sigma, \delta, q, \emptyset)\\
\delta(r,a)=\emptyset, \forall r, a
\end{gather*}

Note: earlier we only say $F \subset Q$,
so $F$ can be $\emptyset$

\item For the other three situations
  \begin{equation*}
    \begin{split}
& R=R_1\cup R_2\\
& R=R_1 \circ R_2\\
& R=R_1^*
\end{split}
\end{equation*}
we use the proof in NFAs
\item We will see details by an example below
\end{itemize}\end{frame}

\begin{frame}[allowframebreaks] \frametitle{Example:
Fig 1.57    $(ab\cup a)^*$}

\begin{tabular}{rc}
  $a$ &
  \begin{tikzpicture}[scale=0.7, every node/.style={scale=0.7}]
\node[state, initial] (q1) {};
\node[state, accepting, right of=q1] (q2) {};

\draw (q1) edge[above] node{$a$} (q2);
\end{tikzpicture}
  \\
  $b$ &
  \begin{tikzpicture}[scale=0.7, every node/.style={scale=0.7}]
\node[state, initial] (q1) {};
\node[state, accepting, right of=q1] (q2) {};

\draw (q1) edge[above] node{$b$} (q2);
\end{tikzpicture}
  \\
  $ab$
      &
  \begin{tikzpicture}[scale=0.7, every node/.style={scale=0.7}]
    \node[state, initial] (q1) {};
    \node[state, right of=q1] (q2) {};
\node[state, right of=q2] (q3) {};        
\node[state, accepting, right of=q3] (q4) {};

\draw (q1) edge[above] node{$a$} (q2);
\draw (q2) edge[above] node{$\epsilon$} (q3);
\draw (q3) edge[above] node{$b$} (q4);
\end{tikzpicture}
  \\
  $ab \cup a$
      &
\begin{tikzpicture}[scale=0.7, every node/.style={scale=0.7}]
    \node[state, initial] (q0) {};          
    \node[state, above right of =q0, yshift=-1.4cm] (q1) {};
    \node[state, right of=q1] (q2) {};
\node[state, right of=q2] (q3) {};        
\node[state, accepting, right of=q3] (q4) {};
\node[state, below right of =q0, yshift=1.4cm] (q5) {};
\node[state, accepting, right of=q5] (q6) {};

\draw (q1) edge[above] node{$a$} (q2);
\draw (q2) edge[above] node{$\epsilon$} (q3);
\draw (q3) edge[above] node{$b$} (q4);

\draw (q5) edge[above] node{$a$} (q6);
\draw (q0) edge[above] node{$\epsilon$} (q1);
\draw (q0) edge[below] node{$\epsilon$} (q5);
\end{tikzpicture}
\end{tabular}


 $(ab \cup a)^*$
        \begin{tikzpicture}[scale=0.7, every node/.style={scale=0.7}]
    \node[state, initial, accepting] (q00) {};                    
    \node[state, right of=q00] (q0) {};          
    \node[state, above right of =q0, yshift=-1.45cm] (q1) {};
    \node[state, right of=q1] (q2) {};
\node[state, right of=q2] (q3) {};        
\node[state, accepting, right of=q3] (q4) {};
\node[state, below right of =q0, yshift=1.45cm] (q5) {};
\node[state, accepting, right of=q5] (q6) {};

\draw (q00) edge[above] node{$\epsilon$} (q0);
\draw (q1) edge[above] node{$a$} (q2);
\draw (q2) edge[above] node{$\epsilon$} (q3);
\draw (q3) edge[above] node{$b$} (q4);

\draw (q5) edge[above] node{$a$} (q6);
\draw (q0) edge[above] node{$\epsilon$} (q1);
\draw (q0) edge[below] node{$\epsilon$} (q5);
\draw (q4) edge[bend right=35, above] node{$\epsilon$} (q0);
\draw (q6) edge[bend left=45, below] node{$\epsilon$} (q0);
\end{tikzpicture}

\begin{itemize}
\item Such a construction usually does not give an
  NFA with the smallest number of
states
\item This example $\Rightarrow$ can be by 2 states
  \begin{equation*}
    (\{q_1, q_2\}, \{a,b\}, \delta, q_1, \{q_1\})
  \end{equation*}
  \begin{center}
  \begin{tabular}{c|ccc}
& a & b & $\epsilon$\\ \hline
$q_1$ & $\{q_1,q_2\}$ & $\emptyset$ & $\emptyset$\\
$q_2$ & $\emptyset$ & $\{q_1\}$ &$\emptyset$
  \end{tabular}
\end{center}
\end{itemize}\end{frame}
\end{document}
