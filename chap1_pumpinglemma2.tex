\input{header}

\AtBeginSection[]
{
	\begin{frame}<beamer>
		\frametitle{Outline}
		\tableofcontents[current,currentsubsection]
	\end{frame}
}

\begin{document}

\begin{frame}[allowframebreaks] \frametitle{The use of pumping lemma}
  \begin{itemize}
    
  \item Let's start from the following statement
    \begin{equation*}
      \forall n, 1 + \cdots + n = \frac{n(n+1)}{2}
    \end{equation*}
    What is the opposite statement?
    \begin{equation*}
      \exists n \text{ such that } 1 + \cdots + n \neq  \frac{n(n+1)}{2}
    \end{equation*}
    
  \item Formally, in the pumping lemma
      \begin{center}
regular $\Rightarrow$ some properties    
  \end{center}
  ``Some properties'' are in fact
  \begin{equation*}
\exists p, \{\forall s \in A, |s|\geq p [ 
\exists x,y,z \text{ such that } s = xyz \text{ and }
\end{equation*}
\begin{equation*}
(xy^iz\in A, \forall i \geq 0,
\mbox{ and }
|y| > 0
\mbox{ and }
|xy|
\leq p)]\}
\end{equation*}
\item Opposite statement of the right-hand side
 \begin{equation}
   \begin{split}
& \forall p, \{\exists s \in A, |s|\geq p, 
[\forall x,y,z 
\\
& (
(s = xyz
\text{ and }
|y| > 0
 \mbox{ and }
 |xy|
 \leq p
)
\\
& \quad \rightarrow
\exists i \geq 0, xy^iz\notin A)]\} 
\end{split}
\label{eq:oppositepumping}
 \end{equation}
\item Note that the opposite of
  \begin{equation*}
    A \& B
  \end{equation*}
is
\begin{equation*}
  A \rightarrow \neg B
\end{equation*}
See the truth table

\begin{center}
  \begin{tabular}{cc|cc|cc}
    $A$ & $B$ & $A \& B$ & $\neg(A\& B)$ & $\neg B$ & $A \rightarrow \neg B$ \\ \hline
    0& 0 &0 & 1 & 1& 1\\ 
    0& 1 &0 & 1 & 0& 1\\
    1& 0 &0 & 1 & 1& 1\\
    1& 1 &1 & 0 & 0& 0
  \end{tabular}
\end{center}
\item To prove \eqref{eq:oppositepumping}, the ``$exists$'' part
  is important
\item You can see that we need to choose $s$ and find an $i$
\item About
  \begin{equation*}
    \forall x, y, z, \cdots
  \end{equation*}
  in \eqref{eq:oppositepumping} you can see in examples that we go through
  all possible cases of $x, y, z$
  
% \item Different form

% $\forall p, \exists s \in A, |s|\geq p, 
% \forall xyz$ with

% $  s= xyz,
% |y| > 0,
% |xy|
% \leq p
% $

% $xy^iz\notin A, \exists i \geq 0$


\end{itemize}\end{frame}

\begin{frame}[allowframebreaks] \frametitle{Example 1.75}
  \begin{itemize}
\item $F=\{ww\mid w \in \{0,1\}^*\}$ not regular
\item We choose
  \begin{equation*}
s=0^p 1 0^p 1 \in F
\end{equation*}
\item If
  \begin{equation*}
    s =xyz,
    |xy|
\leq p, |y|>0,
\end{equation*}
then
\begin{equation*}
 y = 0\ldots 0
\end{equation*}
and thus
\begin{equation*}
xy^2z = 0 \ldots 0 1 0^p 1\neq ww
\end{equation*}
\end{itemize}\end{frame} \begin{frame}[allowframebreaks] \frametitle{Example 1.76}
  \begin{itemize}
\item $D=\{1^{n^2}
\mid n \geq 0\}$ not regular
\item We choose
  \begin{equation*}
  s=1^{p^2} \in D
\end{equation*}
\item If 
  \begin{equation*}
s =xyz, |xy |\leq p, |y|>0
\end{equation*}
then 
\begin{equation*}
p^2 < |xy^2z| \leq p^2 +p < (p+1)^2
\end{equation*}
and therefor
\begin{equation*}
  xy^2 z \not in D
\end{equation*}
\end{itemize}\end{frame}

\end{document}
