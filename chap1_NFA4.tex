\input{header}

\AtBeginSection[]
{
	\begin{frame}<beamer>
		\frametitle{Outline}
		\tableofcontents[current,currentsubsection]
	\end{frame}
}

\begin{document}

\begin{frame}[allowframebreaks] \frametitle{Closure under regular operations}
  \begin{itemize}
  \item Recall we define three operations:
    \begin{equation*}
    \cup, \circ, *
  \end{equation*}
\item We will see that by using NFA, the proof is easier
\end{itemize}\end{frame} \begin{frame}[allowframebreaks] \frametitle{Union}
  \begin{itemize}  
\item Given two regular languages $A_1,A_2$ under the {\bf same }
$\Sigma$

\item [] Is $A_1 \cup A_2$ regular?
\item To prove that a language is regular, by
  definition, it should be accepted by one DFA (or an NFA)

  
\item [] We will construct an NFA for $A_1\cup A_2$
\item Assume $A_1$ and $A_2$ are recognized by two NFAs $N_1$
  and $N_2$, respectively

  \begin{center}
  \begin{tabular}{ll}
    $N_1$ & $N_2$\\
 \begin{tikzpicture}[scale=0.5, every node/.style={scale=0.5}]
\node[state, initial] (1) {$q_1$};
\node[state, above right of=1, yshift=-1.4cm] (2) {};
\node[state, below right of=1, yshift=1.4cm] (3) {};
\node[state, right of=1, xshift= 1.4cm] (6) {};
\node[state, accepting, above right of=6, yshift=-1cm] (4) {};
\node[state, accepting, below right of=6, yshift=1cm] (5) {};
\end{tikzpicture}
    & \begin{tikzpicture}[scale=0.5, every node/.style={scale=0.5}]
\node[state, initial] (01) {$q_2$};
\node[state, above right of=01, yshift=-1.4cm] (02) {};
\node[state, below right of=01, yshift=1.2cm] (03) {};
\node[state, right of= 01, xshift= 1cm] (06) {};
\node[state, accepting, above right of=06, yshift=-1cm] (04) {};
\node[state, accepting, below right of=06, yshift=1cm] (05) {};
\node[state, accepting, right of=06, xshift=1.7cm] (04) {};
\end{tikzpicture}
  \end{tabular}
\end{center}

\begin{center}
  \begin{tikzpicture}[scale=0.5, every node/.style={scale=0.5}]
\node[state, initial] (0) {$q_0$};    
\node[state, above right of=0] (1) {$q_1$};
\node[state, above right of=1, yshift=-1.4cm] (2) {};
\node[state, below right of=1, yshift=1.4cm] (3) {};
\node[state, right of=1, xshift= 1.4cm] (6) {};
\node[state, accepting, above right of=6, yshift=-1cm] (4) {};
\node[state, accepting, below right of=6, yshift=1cm] (5) {};
\node[state, below right of=0] (01) {$q_2$};
\node[state, above right of=01, yshift=-1.4cm] (02) {};
\node[state, below right of=01, yshift=1.2cm] (03) {};
\node[state, right of= 01, xshift= 1cm] (06) {};
\node[state, accepting, above right of=06, yshift=-1cm] (04) {};
\node[state, accepting, below right of=06, yshift=1cm] (05) {};
\node[state, accepting, right of=06, xshift=1.7cm] (04) {};

\path (0) edge[above] node {$\epsilon$} (1);
\path (0) edge[below] node {$\epsilon$} (01);
\end{tikzpicture}
\end{center}

\item Formal definition

\item [] Two NFAs:

  \begin{equation*}
    \begin{split}
&N_1 =(Q_1, \Sigma, \delta_1, q_1, F_1) \\
& N_2 =(Q_2, \Sigma, \delta_2, q_2, F_2)
\end{split}
\end{equation*}
\item [] Note for NFA, $\epsilon \notin \Sigma$

\item New NFA

  \begin{equation*}
    \begin{split}
& Q=Q_1 \cup Q_2 \cup \{q_0\}\\
& q=q_0\\
& F=F_1\cup F_2
\end{split}
\end{equation*}
\begin{equation*}
  \delta(q,a)
=
\begin{cases}
  \delta_1(q,a) &  q \in Q_1\\
  \delta_2(q,a) &  q \in Q_2\\
\{q_1,q_2\} & q= q_0 \mbox{ and }  a = \epsilon \\
\emptyset & q = q_0 \mbox{ and } a \neq \epsilon
\end{cases}
\end{equation*}
\item The last case of $\delta$ is easily neglected
\end{itemize}\end{frame} \begin{frame}[allowframebreaks] \frametitle{Closed Under Concatenation}

Given two NFAs 
\begin{center}
  \begin{tabular}{lll}
    $N_1$ & & $N_2$\\
 \begin{tikzpicture}[scale=0.5, every node/.style={scale=0.5}]
\node[state, initial] (1) {};
\node[state, above right of=1, yshift=-1.4cm] (2) {};
\node[state, below right of=1, yshift=1.4cm] (3) {};
\node[state, right of=1, xshift= 1.4cm] (6) {};
\node[state, accepting, above right of=6, yshift=-1cm] (4) {};
\node[state, accepting, below right of=6, yshift=1cm] (5) {};
\end{tikzpicture}
&    & \begin{tikzpicture}[scale=0.5, every node/.style={scale=0.5}]
\node[state, initial] (01) {};
\node[state, above right of=01, yshift=-1.4cm] (02) {};
\node[state, below right of=01, yshift=1.2cm] (03) {};
\node[state, right of= 01, xshift= 1cm] (06) {};
\node[state, accepting, above right of=06, yshift=-1cm] (04) {};
\node[state, accepting, below right of=06, yshift=1cm] (05) {};
\node[state, accepting, right of=06, xshift=1.7cm] (04) {};
\end{tikzpicture}
  \end{tabular}
\end{center}


\begin{itemize}
\item Idea: from any accept state of $N_1$, add an $\epsilon$ link to
  $q_2$ (start state of $N_2$)
\item The new machine:
\end{itemize}

\begin{center}
  \begin{tikzpicture}[scale=0.5, every node/.style={scale=0.5}]
\node[state, initial] (1) {};
\node[state, above right of=1, yshift=-1.4cm] (2) {};
\node[state, below right of=1, yshift=1.4cm] (3) {};
\node[state, right of=1, xshift= 1.4cm] (6) {};
\node[state, above right of=6, yshift=-1cm] (4) {};
\node[state, below right of=6, yshift=1cm] (5) {};
\node[state, right of=6, xshift=2cm] (01) {};
\node[state, above right of=01, yshift=-1.4cm] (02) {};
\node[state, below right of=01, yshift=1.2cm] (03) {};
\node[state, right of= 01, xshift= 1cm] (06) {};
\node[state, accepting, above right of=06, yshift=-1cm] (04) {};
\node[state, accepting, below right of=06, yshift=1cm] (05) {};
\node[state, accepting, right of=06, xshift=1.7cm] (04) {};

\path (4) edge[above] node {$\epsilon$} (01);
\path (5) edge[below] node {$\epsilon$} (01);

\end{tikzpicture}
\end{center}


  \begin{itemize}

\item Formal definition. Given two automata

  \begin{equation*}
    \begin{split}
&(Q_1, \Sigma, \delta_1, q_1, F_1) \\
&(Q_2, \Sigma, \delta_2, q_2, F_2)
\end{split}
\end{equation*}

\item New machine
  \begin{equation*}
    \begin{split}
& Q=Q_1\cup Q_2 \\
& q_0 = q_1 \\
& F=F_2 \\
\end{split}
\end{equation*}
$\delta$ function:
\begin{equation*}
  \delta(q,a)=
  \begin{cases}
    \delta_1(q,a) & q \in Q_1\backslash F_1\\
\delta_2(q,a) & q \in Q_2\\
\delta_1(q,\epsilon) \cup \{q_2\} & q \in F_1, a =\epsilon\\
\delta_1(q,a) & q \in F_1, a \neq \epsilon
  \end{cases}
\end{equation*}
\end{itemize}
\end{frame}

\begin{frame}[allowframebreaks] \frametitle{Closed under star}

 \begin{itemize}
 \item Given the following machine
   \begin{center}
 \begin{tikzpicture}[scale=0.5, every node/.style={scale=0.5}]
\node[state, initial] (1) {};
\node[state, above right of=1, yshift=-1.4cm] (2) {};
\node[state, below right of=1, yshift=1.4cm] (3) {};
\node[state, right of=1, xshift= 1.4cm] (6) {};
\node[state, accepting, above right of=6, yshift=-1cm] (4) {};
\node[state, accepting, below right of=6, yshift=1cm] (5) {};
\end{tikzpicture}
\end{center}
\item Recall the star operation is defined as
  follows
  \begin{equation*}
A^*=\{x_1 \cdots x_k \mid
k \geq 0, x_i \in A\}
\end{equation*}
\item How about the following diagram
  \begin{center}
\begin{tikzpicture}[scale=0.5, every node/.style={scale=0.5}]
\node[state, initial] (1) {};
\node[state, above right of=1, yshift=-1.4cm] (2) {};
\node[state, below right of=1, yshift=1.4cm] (3) {};
\node[state, right of=1, xshift= 1.4cm] (6) {};
\node[state, above right of=6, accepting, yshift=-1cm] (4) {};
\node[state, below right of=6, accepting, yshift=1cm] (5) {};

\path (4) edge[bend right=35, above] node {$\epsilon$} (1);
\path (5) edge[bend left=35, below] node {$\epsilon$} (1);
\end{tikzpicture}
\end{center}
\item The problem is that $\epsilon$ may not be accepted
\item How about making the start state an accepting one

  \begin{center}
  \begin{tikzpicture}[scale=0.5, every node/.style={scale=0.5}]
\node[state, accepting, initial] (1) {};
\node[state, above right of=1, yshift=-1.4cm] (2) {};
\node[state, below right of=1, yshift=1.4cm] (3) {};
\node[state, right of=1, xshift= 1.4cm] (6) {};
\node[state, above right of=6, accepting, yshift=-1cm] (4) {};
\node[state, below right of=6, accepting, yshift=1cm] (5) {};

\path (4) edge[bend right=35, above] node {$\epsilon$} (1);
\path (5) edge[bend left=35, below] node {$\epsilon$} (1);
\end{tikzpicture}
\end{center}
\item This may make the machine to accept strings not in
  $A$
\item A correct setting
  
  \begin{center}
\begin{tikzpicture}[scale=0.5, every node/.style={scale=0.5}]
\node[state, initial, accepting] (0) {};  
\node[state, right of=0] (1) {};
\node[state, above right of=1, yshift=-1.4cm] (2) {};
\node[state, below right of=1, yshift=1.4cm] (3) {};
\node[state, right of=1, xshift= 1.4cm] (6) {};
\node[state, above right of=6, accepting, yshift=-1cm] (4) {};
\node[state, below right of=6, accepting, yshift=1cm] (5) {};

\path (4) edge[bend right=35, above] node {$\epsilon$} (1);
\path (5) edge[bend left=35, below] node {$\epsilon$} (1);
\path (0) edge[above] node {$\epsilon$} (1);
\end{tikzpicture}
\end{center}

\item Formal definition
\item Given the machine
  \begin{equation*}
 (Q_1, \Sigma, \delta_1, q_1, F_1) 
\end{equation*}
\item New machine:
  \begin{equation*}
    \begin{split}
& Q=Q_1\cup \{q_0\} \\
& q_0: \text{ new start state}\\
& F=F_1 \cup \{q_0\}
\end{split}
\end{equation*}
\begin{equation*}
  \delta(q,a)
=
\begin{cases}
  \delta_1(q,a) & q \in Q_1 \backslash F_1\\
\delta_1(q,a) \cup \{q_1\}& q \in F_1, a =
\epsilon\\
\delta_1(q,a) & q \in F_1, a \neq \epsilon\\
\{q_1\} & q = q_0, a = \epsilon\\
\emptyset & q = q_0, a \neq \epsilon
\end{cases}
\end{equation*}
\end{itemize}\end{frame}
\end{document}
