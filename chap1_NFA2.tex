\input{header}

\AtBeginSection[]
{
	\begin{frame}<beamer>
		\frametitle{Outline}
		\tableofcontents[current,currentsubsection]
	\end{frame}
}

\begin{document}

\begin{frame}[allowframebreaks] \frametitle{Example 1.33}
\begin{itemize}
\item Consider the following figure
    \begin{center}
\begin{tikzpicture}
\node[state, initial] (0) {};
\node[state, accepting, right of=0] (1) {};
\node[state, right of=1] (2) {};
\node[state, accepting, below of=1] (3) {};
\node[state, right of=3] (4) {};
\node[state, below right of=3] (5) {};

\draw (0) edge[above] node{$\epsilon$} (1)
(0) edge[below] node{$\epsilon$} (3)
(1) edge[bend right, below] node{$0$} (2)
(2) edge[bend right, above] node{$0$} (1)
(3) edge[above] node{$0$} (4)
(4) edge[above] node{$0$} (5)
(5) edge[above] node{$0$} (3);
\end{tikzpicture}
\end{center}
\item For this language, $\Sigma=\{0\}$.
  This is called unary alphabets

\item What is the language?
  \begin{equation*}
\{0^k\mid  k \text{ multiples of 2 or 3}\}
\end{equation*}
\end{itemize}\end{frame} \begin{frame}[allowframebreaks] \frametitle{Example 1.35}
  \begin{itemize}
  \item Fig 1.36
  \begin{center}
\begin{tikzpicture}
\node[state, initial, accepting] (q1) {$q_1$};
\node[state, below left of=q1] (q2) {$q_2$};
\node[state, below right of=q1] (q3) {$q_3$};

\draw (q1) edge[left] node{$b$} (q2)
(q2) edge[loop left, left] node{$a$} (q2)
(q1) edge[bend right, left] node{$\epsilon$} (q3)
(q2) edge[below] node{$a, b$} (q3)
(q3) edge[bend right, right] node{$a$} (q1);
\end{tikzpicture}
\end{center}
    
\item Accept

$\epsilon$, $a$, $baba$, $baa$ can be accepted
\item But babba is rejected
\item [] See the tree below

  \begin{center}
\begin{tikzpicture} [level distance=30pt]% grow=right
  \node  {$q_1$}
  child [grow=right] {node  {$q_3$} edge from parent[draw=none]}
  child [grow=down] {node  {$q_2$}
    child {node {$q_2$}
      child {node  {$q_3$}
        child [grow=left] {node (4) {} edge from parent[draw=none]
          child [grow=up] {node (3) {} edge from parent[draw=none]
            child [grow=up] {node (2) {} edge from parent[draw=none]
              child [grow=up] {node (1) {} edge from parent[draw=none]
              }
            }
          }
        }
      }
    }
  };
  \path (1) -- (2) node [midway] {b};
  \path (2) -- (3) node [midway] {a};
  \path (3) -- (4) node [midway] {b};
\end{tikzpicture}
\end{center}
\item This example is later used to illustrate the
  procedure for converting NFA to DFA
\end{itemize}\end{frame}

\begin{frame}[allowframebreaks] \frametitle{Definition: NFA}
  \begin{itemize}
\item $(Q, \Sigma, \delta, q_0, F)$
\item $\delta$: $Q \times \Sigma_\epsilon \rightarrow
P(Q)$

\item [] $P(Q)$: all possible subsets of $Q$
\item $\Sigma_\epsilon
=\Sigma \cup \{\epsilon\}$
\item $P(Q)$: power set of $Q$

\item [] ``power'': all $2^{|Q|}$ combinations 

\begin{equation*}
  Q = \{1,2,3\}
\end{equation*}
\begin{equation*}
  \begin{split}
&  P(Q)= \\
&  \{\emptyset,
\{1\}, \{2\},
\{3\},
\{1,2\},
\{1,3\},
\{2,3\},
\{1,2,3\}\}
\end{split}
\end{equation*}
\end{itemize}\end{frame} \begin{frame}[allowframebreaks] \frametitle{Example 1.38}
  \begin{center}
\begin{tikzpicture}
\node[state, initial] (q1) {$q_1$};
\node[state, right of=q1] (q2) {$q_2$};
\node[state, right of=q2] (q3) {$q_3$};
\node[state, accepting, right of=q3] (q4) {$q_4$};

\draw (q1) edge[above] node{$1$} (q2)
(q1) edge[loop above, left] node{$0,1$} (q1)
(q2) edge[above] node{$0, \epsilon$} (q3)
(q3) edge[above] node{$1$} (q4)
(q4) edge[loop above, right] node{$0,1$} (q4);
\end{tikzpicture}
\end{center}

  \begin{itemize}
\item $Q=\{q_1,\ldots, q_4\}$
\item $\Sigma = \{0,1\}$
\item Start state: $q_1$
\item $F=\{q_4\}$
\item $\delta$:

  \begin{center}
  \begin{tabular}{c|ccc}
& 0 & 1 & $\epsilon$\\ 
\hline
$q_1$ & \{$q_1$\} & $\{q_1, q_2\}$ & $\emptyset$\\
$q_2$ & \{$q_3$\} & $\emptyset$ & \{$q_3$\}\\
$q_3$ & $\emptyset$ & $\{q_4\}$ & $\emptyset$\\
$q_4$ & \{$q_4$\} & $\{q_4\}$ & $\emptyset$
  \end{tabular}
\end{center}
\item Note that DFA does not allow $\emptyset$ 
\end{itemize}\end{frame}

\begin{frame}[allowframebreaks] \frametitle{$N$ accepts $w$}
  \begin{itemize}
\item First we have that $w$ can be written as 
  \begin{equation*}
    w = y_1 \ldots y_m
  \end{equation*}
where $y_i \in \Sigma_{\alert{\epsilon}}$
\item A sequence $r_0 \ldots r_m$ such that

  \begin{enumerate}
  \item $r_0=q_0$
  \item $r_{i+1}\in \delta(r_i,y_{i+1})$
  \item $r_m\in F$
  \end{enumerate}
\item So $m$ may not be the original length (as $y_i$ may be
  $\epsilon$)
\end{itemize}\end{frame}

\end{document}
