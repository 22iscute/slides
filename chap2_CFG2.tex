\input{header}

\AtBeginSubsection[]
{
	\begin{frame}<beamer>
		\frametitle{Outline}
		\tableofcontents[current,currentsubsection]
	\end{frame}
}

\begin{document}

\begin{frame}[allowframebreaks] \frametitle{Example 2.4}
  \begin{itemize}
  \item The following CFG handles mathematical expressions
\item $G_4=
(V, \Sigma, R, \langle \text{expr}\rangle)$

\item [] $V=\{\langle \text{expr}\rangle,\langle \text{term}\rangle,\langle \text{factor}\rangle\}
$

\item [] $\Sigma=\{a,+,\times,(,)\}$
\item [] $R$:
\begin{eqnarray*}
&& \langle \text{expr}\rangle \rightarrow \langle \text{expr}\rangle+\langle \text{term}\rangle\mid\langle \text{term}\rangle\\
&& \langle \text{term}\rangle  \rightarrow \langle \text{term}\rangle\times\langle \text{factor}\rangle\mid 
\langle \text{factor}\rangle\\
&& \langle \text{factor}\rangle\rightarrow (\langle \text{expr}\rangle)\mid a
\end{eqnarray*}

Fig 2.5: check
$a+a\times a$

\framebreak

\scalebox{0.85}{
  \begin{forest}
for tree={
    parent anchor=south,
    child anchor=north,
    if n children=0{
      font=\itshape,
      tier=terminal,
    }{},
  }    
  [E
   [E 
    [T
     [F
      [a]
     ]
    ]
   ]
   [+]
   [T
    [T
     [F
      [a]
     ]
    ]
    [$\times$]
    [F
     [a
     ]
    ]
   ]
  ]
\end{forest}
}

check $(a+a) \times a$

\framebreak

\scalebox{0.55}{
  \begin{forest}
for tree={
    parent anchor=south,
    child anchor=north,
    if n children=0{
      font=\itshape,
      tier=terminal,
    }{},
  }    
  [E
   [T
    [T
     [F
     [[(]]
      [E
       [E
        [T
         [F
          [a]
         ]
        ]
       ]
       [+]
       [T
        [F
         [a]
        ]
       ]
      ]
      [[)]]
     ]
    ]
   [$\times$]
   [F
    [a]
   ]
  ]
 ]
\end{forest}
}
\end{itemize}\end{frame} \begin{frame}[allowframebreaks] \frametitle{Design Grammars}
  \begin{itemize}
\item $\{0^n 1^n \mid n\geq 0\}
\cup \{1^n 0^n \mid n \geq 0\}$
\begin{equation*}
  \begin{split}
& S_1 \rightarrow 0 S_1 1 \mid \epsilon \\
& S_2 \rightarrow 1 S_2 0 \mid \epsilon \\
& S \rightarrow S_1 \mid S_2
\end{split}
\end{equation*}
\item If CFG regular $\Rightarrow $
by DFA

\begin{center}
  \begin{tabular}{l}
$R_i \rightarrow a R_j$ if $\delta(q_i,a)
= q_j$ \\
$R_i \rightarrow \epsilon$ if $q_i \in F$
  \end{tabular}
\end{center}
\item We see the main difference between CFG and DFA

  
\item [] CFG: a rule can be like
  \begin{equation*}
  R_i \rightarrow a R_j b
\end{equation*}
\item [] DFA: a rule can only be
  \begin{equation*}
  R_i \rightarrow a R_j,
\end{equation*}
where we treat each $R_i$ as a state and let
\begin{equation*}
  \delta(R_i, a) = R_j
\end{equation*}

\end{itemize}\end{frame} \begin{frame}[allowframebreaks] \frametitle{Ambiguity}
  \begin{itemize}
\item The same string but obtained in different ways
\item For the example of mathematical expressions
  discussed earlier, what if we consider the following
  rules?
  \begin{gather*}
    \langle \text{expr}\rangle\rightarrow \langle \text{expr}\rangle+\langle \text{expr}\rangle\mid \\
\langle \text{expr}\rangle\times\langle \text{expr}\rangle\mid
    (\langle \text{expr}\rangle)|a
\end{gather*}
We see the following ways to parse
\begin{equation*}
a + a \times a
\end{equation*}
\item Fig 2.6

  \begin{center}
  \begin{tabular}{ccc}
\scalebox{0.85}{
  \begin{forest}
for tree={
    parent anchor=south,
    child anchor=north,
    if n children=0{
      font=\itshape,
      tier=terminal,
    }{},
  }    
  [E
   [E 
    [E
     [a]
    ]
    [+]
    [E
     [a]
    ]
   ]
   [$\times$]
   [E
    [a
    ]
   ]
  ]
\end{forest}
}
&&
\scalebox{0.85}{
  \begin{forest}
for tree={
    parent anchor=south,
    child anchor=north,
    if n children=0{
      font=\itshape,
      tier=terminal,
    }{},
  }    
  [E
   [E 
    [a]
   ]
   [+]
   [E
    [E
     [a]
    ]
    [$\times$]
    [E
     [a
     ]
    ]
   ]
  ]
\end{forest}
}
  \end{tabular}
\end{center}

\item This CFG does not give the precedence relation
\item We want that $a \times a$ is done first
\item By the more complicated CFG earlier,
  the parsing is unambiguous
\item Question: how to formally define the ambiguity?
\item We need to define ``leftmost derivation'' first
\end{itemize}\end{frame} \begin{frame}[allowframebreaks] \frametitle{Leftmost derivation}
  \begin{itemize}
  \item Even for an unambiguous CFG we may
    have
    \begin{center}
the same parse tree, but different derivations
\end{center}
For the CFG discussed earlier we can do
\begin{eqnarray*}
  \langle \text{expr}\rangle& \Rightarrow& \langle \text{expr}\rangle+\langle \text{term}\rangle\\
& \Rightarrow & \langle \text{expr}\rangle+\langle \text{term}\rangle\times\langle \text{factor}\rangle
\end{eqnarray*}
where the second part is expanded. On the other hand, we
can handle the first one first.
\begin{eqnarray*}
\langle \text{expr}\rangle & \Rightarrow & \langle \text{expr}\rangle + \langle \text{term}\rangle\\
& \Rightarrow & \langle \text{term}\rangle + \langle \text{term}\rangle
\end{eqnarray*}
\item This is not considered ambiguous
\item Definition of leftmost derivation:
  at every step the leftmost remaining variable is the one replaced.
\end{itemize}\end{frame} \begin{frame}[allowframebreaks] \frametitle{Formal definition of ambiguity}
  \begin{itemize}
\item $w$ ambiguous if there exist
  \begin{center}
two  leftmost derivations
\end{center}
\item Some context-free languages
can be generated by ambiguous \& unambiguous 
grammars
\item We say a CFG is inherently ambiguous
if it only has ambiguous grammars

\item [] See prob 2.29 in the textbook. Details not given here.

\end{itemize}\end{frame}


\end{document}
