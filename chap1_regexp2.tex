\input{header}

\AtBeginSection[]
{
	\begin{frame}<beamer>
		\frametitle{Outline}
		\tableofcontents[current,currentsubsection]
	\end{frame}
}

\begin{document}

\begin{frame}[allowframebreaks] \frametitle{Example:
Fig 1.57    $(ab\cup a)^*$}

\begin{tabular}{rc}
  $a$ &
  \begin{tikzpicture}[scale=0.7, every node/.style={scale=0.7}]
\node[state, initial] (q1) {};
\node[state, accepting, right of=q1] (q2) {};

\draw (q1) edge[above] node{$a$} (q2);
\end{tikzpicture}
  \\
  $b$ &
  \begin{tikzpicture}[scale=0.7, every node/.style={scale=0.7}]
\node[state, initial] (q1) {};
\node[state, accepting, right of=q1] (q2) {};

\draw (q1) edge[above] node{$b$} (q2);
\end{tikzpicture}
  \\
  $ab$
      &
  \begin{tikzpicture}[scale=0.7, every node/.style={scale=0.7}]
    \node[state, initial] (q1) {};
    \node[state, right of=q1] (q2) {};
\node[state, right of=q2] (q3) {};        
\node[state, accepting, right of=q3] (q4) {};

\draw (q1) edge[above] node{$a$} (q2);
\draw (q2) edge[above] node{$\epsilon$} (q3);
\draw (q3) edge[above] node{$b$} (q4);
\end{tikzpicture}
  \\
  $ab \cup a$
      &
\begin{tikzpicture}[scale=0.7, every node/.style={scale=0.7}]
    \node[state, initial] (q0) {};          
    \node[state, above right of =q0, yshift=-1.4cm] (q1) {};
    \node[state, right of=q1] (q2) {};
\node[state, right of=q2] (q3) {};        
\node[state, accepting, right of=q3] (q4) {};
\node[state, below right of =q0, yshift=1.4cm] (q5) {};
\node[state, accepting, right of=q5] (q6) {};

\draw (q1) edge[above] node{$a$} (q2);
\draw (q2) edge[above] node{$\epsilon$} (q3);
\draw (q3) edge[above] node{$b$} (q4);

\draw (q5) edge[above] node{$a$} (q6);
\draw (q0) edge[above] node{$\epsilon$} (q1);
\draw (q0) edge[below] node{$\epsilon$} (q5);
\end{tikzpicture}
\end{tabular}


 $(ab \cup a)^*$
        \begin{tikzpicture}[scale=0.7, every node/.style={scale=0.7}]
    \node[state, initial, accepting] (q00) {};                    
    \node[state, right of=q00] (q0) {};          
    \node[state, above right of =q0, yshift=-1.45cm] (q1) {};
    \node[state, right of=q1] (q2) {};
\node[state, right of=q2] (q3) {};        
\node[state, accepting, right of=q3] (q4) {};
\node[state, below right of =q0, yshift=1.45cm] (q5) {};
\node[state, accepting, right of=q5] (q6) {};

\draw (q00) edge[above] node{$\epsilon$} (q0);
\draw (q1) edge[above] node{$a$} (q2);
\draw (q2) edge[above] node{$\epsilon$} (q3);
\draw (q3) edge[above] node{$b$} (q4);

\draw (q5) edge[above] node{$a$} (q6);
\draw (q0) edge[above] node{$\epsilon$} (q1);
\draw (q0) edge[below] node{$\epsilon$} (q5);
\draw (q4) edge[bend right=35, above] node{$\epsilon$} (q0);
\draw (q6) edge[bend left=45, below] node{$\epsilon$} (q0);
\end{tikzpicture}

\begin{itemize}
\item Such a construction usually does not give an
  NFA with the smallest number of
states
\item This example $\Rightarrow$ can be by 2 states
  \begin{equation*}
    (\{q_1, q_2\}, \{a,b\}, \delta, q_1, \{q_1\})
  \end{equation*}
  \begin{center}
  \begin{tabular}{c|ccc}
& a & b & $\epsilon$\\ \hline
$q_1$ & $\{q_1,q_2\}$ & $\emptyset$ & $\emptyset$\\
$q_2$ & $\emptyset$ & $\{q_1\}$ &$\emptyset$
  \end{tabular}
\end{center}
\end{itemize}\end{frame}

\begin{frame}[allowframebreaks] \frametitle{Regular $\Rightarrow$  a regular
    expression}
  \begin{itemize} 
  \item That is, if given an NFA, how can we convert it to a
    regular expression?
\item Example:

  \begin{center}
    \begin{tikzpicture}
    \node[state, initial] (1) {1};                    
    \node[state, accepting, right of=1] (2) {2};          

    \draw (1) edge[above] node{$b$} (2);
\draw (1) edge[loop above, left] node{$a$} (1);    
\draw (2) edge[loop above, right] node{$a,b$} (2);    
\end{tikzpicture}
  \end{center}
  
  Quickly we see that this corresponds to
  \begin{equation*}
  a^*b(a\cup b)^*
\end{equation*}
\item Example 1.68

\item [] Consider the following DFA

  \begin{center}
    \begin{tikzpicture}
    \node[state, initial] (1) {1};                    
    \node[state, accepting, right of=1] (2) {2};
    \node[state, accepting, below right of=1] (3) {3};              

    \draw (1) edge[bend left=10, above] node{$a$} (2);
    \draw (2) edge[bend left=10, below] node{$a$} (1);
    \draw (1) edge[bend left=10, right] node{$b$} (3);
    \draw (3) edge[bend left=10, below] node{$b$} (1);    
    
    \draw (2) edge[loop above, right] node{$b$} (2);
    \draw (3) edge[right] node{$a$} (2);        
\end{tikzpicture}
  \end{center}
\item It is not that easy to directly see what the regular
  expression is
\item We need a procedure shown below
\item First, add a start and an accept states
    \begin{center}
      \begin{tikzpicture}
        \node[state, initial] (s) {$s$};
    
    \node[state, above right of=s] (1) {1};                    
    \node[state, right of=1] (2) {2};
    \node[state, below right of=1] (3) {3};              
        \node[state, accepting, below right of=2] (a) {$a$};        

    \draw (1) edge[bend left=10, above] node{$a$} (2);
    \draw (2) edge[bend left=10, below] node{$a$} (1);
    \draw (1) edge[bend left=10, right] node{$b$} (3);
    \draw (3) edge[bend left=10, below] node{$b$} (1);    
    
    \draw (2) edge[loop above, right] node{$b$} (2);
    \draw (3) edge[right] node{$a$} (2);        

    \draw (s) edge[above] node{$\epsilon$} (1);        

    \draw (2) edge[above] node{$\epsilon$} (a);        
    \draw (3) edge[below] node{$\epsilon$} (a);

  \end{tikzpicture}
  \end{center}

\item This generates a
\alert{generalized NFA}
\item Our procedure is
  
  \begin{center}
  DFA $\rightarrow$ GNFA $\rightarrow$
regular expression
\end{center}
\item Remove state 1
\begin{center}
\begin{tikzpicture}
  \node[state, initial] (s) {$s$};
    
  \node[state, above right of=s] (2) {2};
  \node[state, below right of=s] (3) {3};              
  \node[state, accepting, below right of=2, xshift=0.8cm] (a) {$a$};        

  \draw (s) edge[above] node{$a$} (2);
  \draw (s) edge[below] node{$b$} (3);        
  \draw (3) edge[bend right=10, right] node{$ba \cup a$} (2);
  \draw (2) edge[bend right=10, left] node{$ab$} (3);
  \draw (2) edge[loop right, right] node{$aa\cup b$} (2);
  \draw (3) edge[loop right, right] node{$bb$} (3);  
    \draw (2) edge[above] node{$\epsilon$} (a);        
    \draw (3) edge[below] node{$\epsilon$} (a);
  \end{tikzpicture}
  \end{center}

\item Example: the link
  \begin{equation*}
    3 \rightarrow 2
  \end{equation*}
  \begin{center}
    \begin{tikzpicture}
  \node[state] (3) {3};
  \node[state, below right of=3] (1) {1};
  \node[state, right of=3, xshift=1cm] (2) {2};                

  \draw (3) edge[above] node{$a$} (2);
  \draw (3) edge[left] node{$b$} (1);
  \draw (1) edge[right] node{$a$} (2);          
    \end{tikzpicture}
  \end{center}
\item Thus $ba \cup a$
  
\item Remove state 2
  \begin{center}
\begin{tikzpicture}
  \node[state, initial] (s) {$s$};
  \node[state, below right of=s] (3) {3};              
  \node[state, accepting, right of=s, xshift=0.8cm] (a) {$a$};        

  \draw (s) edge[above] node{$a(aa\cup b)^*$} (a);          
  \draw (s) edge[left] node{$a(aa\cup b)^* ab \cup b$} (3);        
  \draw (3) edge[loop below, below] node{$(ba \cup a)(aa\cup b)^*ab \cup bb$} (3);  
    \draw (3) edge[right] node{$(ba \cup a)(aa\cup b)^* \cup \epsilon$} (a);
  \end{tikzpicture}
\end{center}

\item Example:
  \begin{equation*}
    s \rightarrow 3
  \end{equation*}

  \begin{center}
    \begin{tikzpicture}
  \node[state] (s) {$s$};
  \node[state, below right of=s] (2) {2};
  \node[state, right of=s, xshift=1cm] (3) {3};                

  \draw (s) edge[above] node{$b$} (3);
  \draw (s) edge[left] node{$a$} (2);
  \draw (2) edge[right] node{$ab$} (3);
  \draw (2) edge[loop below] node{$aa \cup b$} (2);            
    \end{tikzpicture}
  \end{center}
\item Thus $a (aa \cup b)^*ab  \cup b$
\item Here we need to handle
  \begin{equation*}
    2 \xrightarrow{aa \cup b} 2
  \end{equation*}

\item Remove state 3

  \begin{center}
\begin{tikzpicture}
  \node[state, initial] (s) {$s$};
  \node[state, accepting, right of=s, xshift=3cm] (a) {$a$};        

  \draw (s) edge[below] node{} (a);          
  \end{tikzpicture}
\end{center}
\begin{equation*}
  \begin{split}
& (a(aa\cup b)^* ab \cup b)
((ba \cup a)(aa \cup b)^* ab \cup bb)^*
\\
&((ba \cup a)
    (aa\cup b)^* \cup \epsilon) \cup a(aa \cup b)^*
  \end{split}
\end{equation*}
\item We will formally explain the procedure
\end{itemize}
\end{frame}

\end{document}
